\documentclass[12pt,letterpaper]{article}
\usepackage[pdftex]{graphicx}
\usepackage{alltt}
\usepackage[margin=1in]{geometry}
\usepackage{amsmath, amsthm, amssymb}
\usepackage{verbatim}
\usepackage{ragged2e}
\usepackage{enumitem}
\usepackage{xfrac}
\setlist{parsep=0pt,listparindent=\parindent}
\setlength{\RaggedRightParindent}{\parindent}
\newcommand{\degree}{\ensuremath{^\circ}}
\usepackage{accents}
\let\thinbar\bar
\newcommand\thickbar[1]{\accentset{\rule{.4em}{.8pt}}{#1}}
\let\bar\thickbar
\usepackage{standalone}
%\usepackage{hyperref}
\usepackage{venndiagram}
\usepackage{mathtools}
\usepackage{bm}
\usepackage[mathscr]{eucal}
\usepackage{cleveref}

\newcommand{\R}{\ensuremath{\mathbb{R}}}
\newcommand{\Q}{\ensuremath{\mathbb{Q}}}
\newcommand{\Z}{\ensuremath{\mathbb{Z}}}
\newcommand{\card}{\ensuremath{\operatorname{card }}}
\newcommand{\Mod}[1]{\ensuremath{\;\;(\operatorname{mod}\;#1)}}
\DeclarePairedDelimiter{\ceil}{\lceil}{\rceil}
\DeclarePairedDelimiter{\floor}{\lfloor}{\rfloor}
\DeclarePairedDelimiter\abs{\lvert}{\rvert}
\DeclarePairedDelimiter\norm{\lVert}{\rVert}
\DeclarePairedDelimiter\angled{\langle}{\rangle}
\newcommand\inv[1]{\ensuremath{{#1}^{-1}}}
% \newcommand\arrowdef{\ensuremath{\stackrel{\mathclap{\normalfont\mbox{def}}}{\iff}}}
\newcommand\arrowdef{\mathrel{\overset{\makebox[0pt]{\mbox{\normalfont\scriptsize\sffamily def}}}{\iff}}}

\newcommand{\divides}{\bigm|} % http://tex.stackexchange.com/questions/117032/divides-not-divides-and-cardinalities
\newcommand{\ndivides}{%
  \mathrel{\mkern.5mu % small adjustment
    % superimpose \nmid to \big|
    \ooalign{\hidewidth$\big|$\hidewidth\cr$\nmid$\cr}%
  }%
}

\crefname{prop}{proposition}{propositions}
\crefname{cor}{corollary}{corollaries}
\crefname{dfn}{definition}{definitions}

\theoremstyle{plain}
\newtheorem{prop}{Proposition}[section]
\newtheorem{cor}{Corollary}[prop]
\theoremstyle{definition}
\newtheorem{dfn}[prop]{Definition}
\newtheorem{innerProp}{Proposition}[prop]
\theoremstyle{remark}
\newtheorem*{remark}{Remark}

\begin{document}

\section{Introduction}
We define some basic terminology related to Conway's Game of Life played on a torus, with a vertical wall as the starting configuration. We assume boards are of height $k\geq 5$, and specify the width, which may also be referred to as the size. Any given numeric variable is a natural, $\N=\Z_{\geq 0}$, unless specified otherwise.
\begin{dfn}
  A board is a rectangular grid of cells, which are dead or alive. The opposite edges of the grid are connected to form a torus. When each column of a grid consists of cells which are either all alive or all dead, we represent the whole grid by a sequence of 0s and 1s, with 1 representing a column of living cells. Let $\mathbb{B}_w$ be the set of all boards of width $w$, with columns which consist of either all living cells or all dead cells. We allow any height, as mentioned above. Let $C:\mathbb{B}_w\rightarrow \mathbb{B}_w$ map the board to what it is after one application of the rule of Conway's Game of Life to it, and let $C^n$ represent $n$ compositions of $C$, that is, $C^n=\underbrace{C\circ C\circ\cdots \circ C}_{n \text{ times}}$. 
\end{dfn}
\begin{dfn}[Board summation]
  We define the sum of two boards, $\oplus:\B_w\times \B_v\rightarrow \B_{w+v}$, where $B\oplus C$ is defined as the two boards placed adjacent to each other. $010\oplus 010 = 010010$. If we have a sequence ${(B_i)}_{1\leq i\leq n}$ of boards of widths $(w_i)$, we write $\displaystyle\bigoplus_{i=1}^{n}B_i = B_1\oplus B_2\oplus\cdots\oplus B_n$. 
\end{dfn}
\begin{dfn}
  We denote a board consisting of $a$ of the same digit $x$ by $a\cdot x = x\cdot a$. For example, $5\cdot 0=00000$. \\
  More generally, we define $a$ copies of the same board by $a\cdot B = B\cdot a$. Formally, $a\cdot B = \displaystyle\bigoplus_{i=1}^{a}B$, that is, $(10100)\cdot 3 = 101001010010100$. \\
  $\cdot$ binds more tightly than $\oplus$, naturally, and less tightly than the standard operations on integers. 
\end{dfn}

\begin{dfn}\label{mainBoard1Center}
  We define a map, $G_w: \N\rightarrow \B_w$ as follows. For all $2k+1$, $G_{2k+1}(0) = k\cdot 0 \oplus 1 \oplus k\cdot 0$. For all $2k$, $G_{2k}(0) = k-1\cdot 0 \oplus 1 \oplus k\cdot 0$. Then for all $w$, $G_w(s)$ is $s$ applications of $C$ to $G_w(0)$, it is $C^s(G_w(0))$. We say that the $1$ on the board $G_w(0)$ is at the center of the board.
\end{dfn}
\begin{dfn} \label{BoardEvolution}
  A {\bf board evolution} is a function $G:\N \rightarrow \B_w$, for some $w$, such that for all $n>0$, $G(n)=C^n(G(0))$. $G_w$ is an example of a board evolution. Clearly, a board evolution $G$ is defined uniquely by $G(0)$. 
\end{dfn}
\begin{dfn}\label{cycles}
  For any board evolution $G$, the {\bf time to cycle} of a board evolution, $tc(G)$, is the minimum $s$ such that there exists a $t<s$ with $G(s)=G(t)$. The {\bf fuse length}, $fuse(G)$ is $t$. Sometimes we also call the board $G(t)$ the fuse or the fuse board. 
\end{dfn}
\begin{dfn}\label{cluster}
  A {\bf cluster} of size $n$ is a largest possible contiguous segment of 1s, of length $n$,  that is, $0\oplus n\cdot 1\oplus 0$ is a cluster of size $n$. A board contains a cluster of size $n$ if it can be written $A\oplus n\cdot 1\oplus B$, where either $A$ is empty, or $A=A'\oplus 0$ and either $B$ is empty or $B=0\oplus B'$. We often mention clusters in the context of the seen clusters of a board with width $w$ --- the set of all $n$ such that a cluster of length $n$ exists in $G_w(s)$ for some $s$. 
\end{dfn}
\begin{dfn}\label{rotation}
  A {\bf (right) rotation} of a board is defined as would be expected, it is an operation $\curvearrowright: \B_w\times \N\rightarrow \B_w$ and is formally defined $$x_1x_2x_3\dots x_m\curvearrowright n = x_{m-n+1}x_{m-n+2}\dots x_{m-1}x_mx_1x_2\dots x_{m-n}$$ where each $x_i$ represents a living or dead cell, for $0\leq n<m$, all appended together. For $n\geq m$, take the representative between $0$ and $m-1$ of the equivalence class $n\Mod m$. The left rotation is defined similarly.
\end{dfn} %which is a group action of Z_w on the board
\begin{dfn}\label{annihilation}
  A board $B$ or a board evolution $G$ {\bf annihilates} in $s$ steps if $s$ is the minimum number of steps such that $s$ applications of the game of life rule to $B$ yields a board where all the cells are dead, that is, $G(s)$ is a board with all dead cells. We also say a board is annihilited if it contains all dead cells.
\end{dfn}

%%%%%%%%%%%%%%%%%%%%%%%%%%%%%%%%%%%%%%%%%%%%%%%%%%%%%%%%%%%%%%%%%%%%

We proceed to prove some basic propositions which will be fairly universal. We start by showing that any odd board in the board evolution $G_w$ is symmetrical.
\begin{prop}[Symmetry]\label{symmetry}
  For a board of width $w=2n+1$, a board $G_w(s)$ for any $s$ can be written $x_1\oplus x_2\oplus\cdots\oplus x_n\oplus c \oplus x_n\oplus\cdots\oplus x_2\oplus x_1$.
\end{prop}
\begin{proof}
  For any width, $s=0,1$ act as clear base cases. Suppose the hypothesis up to step $s$, we show it for $s+1$. Let $G_w(s)$ be written as above, and let $G_w(s+1)=y_1\oplus y_2\oplus\cdots\oplus y_{2n+1}$. Then the game of life neighborhood of $y_1$ is $x_1,x_1,x_2$, and for $y_n$ it is $x_2,x_1,x_1$, so $y_1=y_{2n+1}$. Now consider an arbitrary $y_i$ with $1<i<n$. $y_i$ has neighborhood $x_{i-1},x_i,x_{i+1}$. The symmetric one, $y_{2n+2-i}$ has the equivalent neighborhood $x_{i+1},x_i,x_{i-1}$. Hence $y_{2n+2-i}=y_i$. Finally, for $y_n$ the neighborhood is $x_{n-1}, x_n, x_c$, likewise for $y_{n+2}$, completing the proof.
\end{proof}

%\begin{prop}\label{StepsToReachEdge}
%  Suppose we have a board $a\cdot 0\oplus 1\oplus X\oplus 1\oplus a\cdot 0$. In $0\leq b\leq a$ steps, it will be $a-b\cdot 0\oplus 1\oplus %Y\oplus 1\oplus a-b\cdot 0$, for some sequences $X,Y$.
%\end{prop}
%\begin{proof}
%  This is immediately clear. After a single step, we will have $a-1$ zeros, a 1, and then unknown material until the symmetric portion, simply by the game of life rules. Applying this iteratively gets the result. We do not proceed further, we're out of 0s to proceed into, the 1s have met at edges.
%\end{proof}

Boards whose width is a power of 2, or one less than a power of 2, are frequently very important, since any single column can be modelled as one of these boards for several steps. We begin by showing that a board of size $2^n-1$ annihilates in $2^{n-1}$ steps, for integers $n\geq 3$.

\begin{prop}\label{Power2} % About boards of width 2^n-1
  Let $w(n)=2^n-1$, to generate the width of the board, and $s(n)=2^{n-1}$ to generate the number of steps for the board to annihilate. Then:\\
  \vspace*{-1.3em}
  \begin{enumerate}[label=(\alph*)]
    \item\label{a} The board $G_{w(n)}(s(n-1))$ will be exactly $G_{w(n-1)}(0)\oplus 1\oplus G_{w(n-1)}(0)$.
    \item\label{b} The board $G_{w(n)}(s(n)-1)$ will be alive at the `edges', in clusters of 3: $111\oplus B\oplus 111$, for some sequence of cells $B$. 
    \item\label{c} The board $G_{w(n)}(s(n))$ will be annihilated.
    \item\label{d} The board $G_{w(n)}(s(n)-1)$ (and therefore all states up to it) arrives at its state without ever interacting at the `edges,' that is, for all odd $W>w(n)$, we have that $G_{W}(s(n)-1)$ is $G_{w(n)}(s(n)-1)$ padded with $(W-w(n))/2$ dead cells on each side: $$G_{W}(s(n)-1) = (W-w(n))/2\cdot 0 \oplus G_{w(n)}(s(n)-1) \oplus (W-w(n))/2\cdot 0$$
    %\item\label{e} The board $G_{w(n)+2}(s(n))$ will be exactly $1\oplus w(n)\cdot 0 \oplus 1$
  \end{enumerate}
\end{prop}
\begin{proof}
We have as a base case that it annihilates on size 3 in 2 steps, the states prior to annihilation are 010 and 111. On 7 it annihiliates in 4 steps. On a board of size 7, the second generation, 0011100 lives to create a new state, which could be modeled as two adjacent boards of size 3 in the start state with a gap of one between them: 0100010. The final state non-annihilated is then 1110111.

We assume the hypothesis up to some $n$, proceeding by strong induction, and prove it for $n+1$

By part~\ref{d}, $G_{w(n+1)}(s(n)-1)$ will be a board $G_{w(n)}(s(n)-1)$ padded with zeros. By part 2, it will be alive at the `edges'. Note that $w(n+1)-w(n)=(2^{n+1}-1-2^{n}+1)/2=2^{n-1}$, so we have
$$G_{w(n+1)}(s(n)-1) = 2^{n-1}\cdot 0 \oplus 111 \oplus B \oplus 111 \oplus 2^{n-1}\cdot 0$$
where $B$ is the proper sequence, not explicitly given. On the next generation, by~\ref{c}, the entire center block $B$ will annihilate. However, the columns left and right of this block will live, that is, this generation $G_{w(n+1)}(s(n-1))$ will be exactly $2^{n-1}-1\cdot 0 \oplus 1 \oplus [W\cdot 0]\oplus 1 \oplus 2^{n-1}-1\cdot 0$. This center block will be of size $W=w(n)=2^n-1$, naturally, the size of the embedded center board. So we have $w(n-1)\cdot 0\oplus 1 \oplus w(n-1)\cdot 0\oplus 0 \oplus w(n-1)\cdot 0\oplus 1\oplus w(n-1)\cdot 0$, for a total width of $w(n+1)$ as desired. This state $G_{w(n+1)}(s(n))$ is exactly in the state it was hypothesized to be in, we have proven part~\ref{a}, it is composed of two `adjacent' boards $G_{w(n)}(0)$.

By part~\ref{d}, these two boards $G_{w(n)}(0)$ will advance through $s(n)-1$ states without interacting, hence $G_{w(n+1)}(s(n+1)-1)$, that is, $G_{w(n+1)}(s(n)+(s(n)-1))$ is exactly $G_{w(n)}(s(n)-1) \oplus 0 \oplus G_{w(n)}(s(n)-1)$. In particular, by~\ref{b}, we have $G_{w(n+1)}(s(n+1)-1) = 111 \oplus B \oplus 1110111 \oplus B \oplus 111$, with $B$ being parts of the board. This proves~\ref{b}. In the next generation, the middle parts $B$ evolve from $G_{w(n)}(s(n)-1)$ and so die by~\ref{c}, clearly every other column also dies. Hence we have~\ref{c}. We have arrived to this state without interaction at the edges, we could grow the board arbitrarily without changing the result up to this step, hence we have~\ref{d}.

%Finally, from the proof of~\ref{a} we see that by removing extra padding 0s we get exactly the result~\ref{e}. We can remove them due to part~\ref{d}. Note that we need not include this part in the induction, we could have proven it separately.
\end{proof}

\begin{cor}\label{SingleWallPairPow2}
  For any board of width $W>2^n$, $G_W(2^k)=a\cdot 0\oplus 1\oplus 2^{k+1}-1\cdot 0\oplus 1\oplus a\cdot 0$, for an $a$ appropriate for the width, for $k$ such that $2^{k+1}<W$. For even board the number of 0s on one side must be adjusted by one.
\end{cor}
\begin{proof}
  This follows directly from parts~\ref{a} and~\ref{d} above, and is simply a convenient reformulation.
\end{proof}

We generalize this corollary to any annihilating board.
\begin{prop}\label{PostAnnihilation}
  Suppose $B_w$ is a board evolution of some starting state $B$ (so $B_w(0)=B$) which annihilates in $s$ steps. Furthermore, suppose that $B_w(s-1)$ is alive by the antipode, that is $B_w(s-1)=1\oplus A\oplus 1$. Let $W\geq w+2$. We define a board evolution in two cases. Write $W=w+k$, then if $k=2a$ is even we define $B_W(0)=a\cdot 0 \oplus B\oplus a\cdot 0$, and if $k=2a+1$ is odd we define $B_w(0)=a\cdot 0\oplus B\oplus a+1\cdot 0$. \\
  Then, $B_W(s)=a-1\cdot 0\oplus 1\oplus w\cdot 0\oplus 1\oplus a-1\cdot 0\oplus \alpha\cdot 0$, where $\alpha$ is 0 if $k$ is even and odd otherwise.
\end{prop}
\begin{proof}
  This is trivial, simply by starting at $B_W(s-1)$ and applying the game of life rule. We know that the entirety of the center $w$ part of the board would annihilate next turn on a smaller board, so all of these cells will be dead next turn. The two 0s closest to the center are born, and there is no other change.
\end{proof}

\noindent We generalize part~\ref{d} above to boards of arbitrary odd width.
\begin{prop}\label{OddBoardFirstEdge} % About first time a board reaches the edge, and embedding it in larger boards
  For any $k$, the board $G_{2k+1}(k)$ (and all boards up to it) arrives to its state without ever interacting at the `edges', that is, for all odd $W=2w+1>2k+1$ we have $G_W(k) = w-k \cdot 0 \oplus G_{2k+1}(k)\oplus w-k\cdot 0$. For all even $W=2w>2k+1$ we have $G_W(k) = w-k-1\cdot 0 \oplus G_{2k+1}(k)\oplus w-k\cdot 0$. \\
  Furthermore, $G_{2k+1}(k)$ is alive at both its `edges', $G_{2k+1}(k)=1\oplus A\oplus 1$, with some unspecified $A$.
\end{prop}
\begin{proof}
  % Suppose $W=2w+1$. Let $W$'s binary representation be $b_nb_{n-1}\dots b_1b_0$, with each $b_i\in\{0,1\}$, and $b_n=1$. Then by~\ref{Power2} $G_W(2^{n-1})$ is the board of size $2^n-1$ embedded in $W$ after $ $
  
  We proceed by induction, this is true for simple base cases $W=3,4,5$. Consider odd $W=2k+1$, and suppose the hypothesis is true up to $k$; we prove it for $k+1$, $W=2k+3$. Then note the board $G_{2k+3}(k)=0\oplus G_{2k+1}(k)\oplus 0$, and in particular $G_{2k+3}(k)=01\oplus A\oplus 10$. The next generation is clearly the same even if the board were widened with any number of 0s padded on any side, and will have 1s at its edges. \\
  In particular, if we add one 0 on the right, we see that the hypothesis holds for even boards.
\end{proof}

\begin{dfn}\label{EvenlySpaced}
  We say a board is {\bf evenly spaced} if it consists of $n$ single living columns separated by $n$ gaps of equal length, that is, if the board is of the form $(1\oplus k\cdot 0)\cdot n$ rotated by any amount:
  \[(1\oplus k\cdot 0)\cdot n \curvearrowright m \]
  for arbitrary $k,n$ and $0\leq m < (k+1)n$. Also, $k$ is the {\bf gap size} of the board. 
\end{dfn}

\begin{prop}\label{Pow2BelowPow2EvenSpacing}
  A board of width $2^n$ at steps $2^{n-1}-2^\alpha$, for $1\leq \alpha\leq n-1$ is evenly spaced, or more precisely,
  \[ G_w(2^{n-1}-2^\alpha)=2^\alpha-1\cdot 0 \oplus (1\oplus 2^\alpha*2-1\cdot 0)\cdot 2^{n-1-\alpha}-1 \oplus 1\oplus 2^\alpha\cdot 0 \]
\end{prop}
\begin{remark}
  Recall that this is a torus, so combining the zeros at the `edges' gets another segment of 0s of size $2^\alpha*2+1$. This makes for a total of $2^{n-1-\alpha}$ gaps and living columns.
\end{remark}
\begin{proof}
  For $\alpha=n-1$ and $\alpha=n-2$ this is clear from the definition of $G_w(0)$ and from~\cref{Power2} respectively. We proceed by decreasing induction, suppose this is true for $\alpha+1$, we show it is true for a $\alpha$, with $\alpha>0$. That is, we have that $G_w(2^{n-1}-2^{\alpha+1})$ is of that form. Consider the board in a further $2^\alpha$ steps, this will get the board $G_w(2^{n-1}-2^\alpha)$. To do this, we simply consider each 1 as a board of width $2^{\alpha+2}$, and see that in $2^\alpha$ steps we will get $2^\alpha-1\cdot 0 \oplus 1\oplus 2^\alpha*2-1\cdot 0\oplus 1\cdot 2^\alpha\cdot 0$ by~\cref{Power2} part~\ref{a}. When we append together all these `mini-boards' of width $2^{\alpha+2}$ we get exactly our result.
\end{proof}

\begin{prop}\label{EvenlySpacedPow2Annihilation}
  An evenly spaced board with a gap size which is one less than a power of two annihilates, specifically, $B = (1\oplus 2^k-1\cdot 0)\cdot n \curvearrowright m$ annihilates in $2^{k-1}$ steps.
\end{prop}
\begin{proof}
  This is clear from~\cref{Power2}. Rotate the board appropriately and treat each living column as $G_{2^k-1}(0)$ separated by a single 0, we get $B=(G_{2^k-1}(0)\oplus 0)\cdot (n-1)\oplus G_{2^k-1}(0)$. In $2^{k-1}-1$ steps we then have a board which is a step away from annihilation.
\end{proof}

\section{Boards of Width Three Modulo Four}
We begin examining boards of width $4k+3$. These evidently have a much more regular behavior than boards of width $4k+1$. 
We seek to prove that for boards of width $4k+2, 4k+3, 4k+4$, the survival, time to cycle, and fuse length will be equal. To do this, we will need some more development. In particular, we need some more information about the evolution of boards with width 3 modulo 4.

\begin{prop}\label{onesSepByOddZeros} %boards 4k+3 are 1s sep by odd (3 mod 4) segments of 0s
  Boards of width $w=4k+3$ after an even number of steps can always be written as single 1s separated by segments of 0s with length 3 modulo 4, except at the edges where it is merely odd. That is, in the following format:
  $$G_w(2s)=b\cdot 0 \oplus (1 \oplus a_1\cdot 0 \oplus 1\oplus a_2\cdot 0\oplus\cdots \oplus 1\oplus a_n\cdot 0) \oplus 1 \oplus b\cdot 0$$
  where each $a_i\equiv 3 \Mod{4}$, and $b$ is odd.
  %Equivalently, due to symmetry and the center being 0,
  %\begin{align*}
  %  G_w(2s)=b\cdot 0 & \oplus (1\oplus a_1\cdot 0 \oplus 1\oplus a_2\cdot 0\oplus\cdots\oplus 1\oplus a_n\cdot 0) \oplus 1\oplus c\cdot 0 \oplus 0\oplus c\cdot 0 \oplus 1 \\
  %                   & \oplus (a_n\cdot 0\oplus 1\oplus a_{n-1}\cdot 0\oplus 1\oplus \cdots\oplus a_1\cdot 0\oplus 1)\oplus b\cdot 0
  %\end{align*}
  %with each $a_i,b,c$ odd, $a_i,c\geq 3$, $b\geq 1$. \\
  On odd steps we have clusters 111 separated by segments of zeros of length $1\Mod 4$, we will not write it explicitly.
\end{prop}
\begin{proof}
  Clearly for any such $w$ with $k\geq 1$, this holds for a base case of steps 0, and for that matter step 2. Suppose it holds up to $2s$, we show it holds at step $2s+2$. The next step $G_w(2s+1)$ is $$b-1\cdot 0\oplus (111\oplus a_1-2\cdot 0 \oplus 111\oplus a_2-2\cdot 0\oplus\cdots\oplus 111 \oplus a_n-2\cdot 0)\oplus 111 \oplus b-1$$ exactly as in the proposition. For each $a_i$, $a_i-2\equiv 1 \Mod 4$. \\
  Then for $G_w(2s+2)$ we break off into two cases. Suppose $b>1$. Then we define a sequence $(a')$ as follows. If $a_1=3$, collect as many terms after $a_2,a_3,\dots,a_i$ such that each term is also 3. Set $a'_1=i*4-1$. Else, $a'_1=a_1-2$. In both these cases, $a'_1\equiv 3 \Mod 4$. Repeat the procedure for $a'_2$ starting from $a_{i+1}$ in the first case, and $a_2$ in the second case. This generates a sequence of length $m$. So, $G_w(2s+2) = b-2\cdot 0\oplus (1\oplus a'_1\cdot 0\oplus 1 \oplus a'_2\cdot 0 \oplus\cdots\oplus 1\oplus a'_m\cdot 0)\oplus 1\oplus b-2\cdot 0$. This satisfies the hypothesis. A note before proceeding to the other case: $m$ is odd due to symmetry,~\cref{symmetry}. If every $a_i=3$, then  \\
  In the other case, suppose $b=1$. Define $(a')$ the same way. Then if $a_1=3$ (which by symmetry implies $a_n=3$), then $G_w(2s+2)= a'_1\cdot 0\oplus (1\oplus a'_2\cdot 0\oplus 1\oplus a'_3\cdot 0\oplus\cdots\oplus 1\oplus a'_{m-1}\cdot 0)\oplus 1\oplus a'_m\cdot 0$. A subtlety to consider: suppose $m=1$. Then the board annihilates. Otherwise, since $m$ is odd, we have an $a'_i$ on either side and at least one in the center, as desired. 
  If $a_1\neq 3$, and so $a_n\neq 3$, we get $G_w(2s+2) = 000\oplus(1\oplus a'_1\cdot 0\oplus 1\oplus a'_2\cdot 0\dots\oplus 1\oplus a'_m\cdot 0)\oplus 1000$. 
\end{proof}

\begin{cor}\label{cluster3mod4cor}
  Boards of width $w=4k+3$ do not produce cluster sizes other than $1,3,6$
\end{cor}
\begin{proof}
  Simply by examining a possible generation $G_w(s)$ as specified above, we see that no other size is possible.
\end{proof}

\begin{cor}\label{3mod4annihilateOnEven}
  If a board of width $w=4k+3$ annihilates, it does so on an even step.
\end{cor}
\begin{proof}
  This is immediate, the only case where annihilation could occur was pointed out, it occurs as a result of an even step.
\end{proof}
\begin{prop}\label{cycleOnEven3mod4}
  If a board of width $w=4k+3$ survives, $tc(G_w)$ will be even, as will $fuse(G_w)$. 
\end{prop}
\begin{proof}
  Suppose $tc(G_w)=s'$ were odd, that is, $2s+1=s'$ and there is a $t'<s'$ such that $G_w(t')=G_w(s')$. First, $t'$ is odd, since if it were even every cluster on $G_w(t')$ would be of size 1, whereas $G_w(2s+1)$ has clusters of size 3 and possibly 6. Let $t'=2t+1$. \\
  We show that $G_w(2t)=G_w(2s)$, which contradicts that $tc(G_w)=s'$, since $s'$ must be the least step with this property. Indeed, this follows almost immediately from~\cref{onesSepByOddZeros}. On every odd step new cells are born, no cells die. Therefore, $G_w(2t+1)$ has a unique predecessor board reachable from $G_w(0)$. We will not write it explicitly, but it is the board reached by letting the center of each cluster of 3 (and treating clusters of 6 as two clusters of 3 in this procedure) be alive, and every other cell be dead.
\end{proof}

\begin{prop}\label{cluster3mod4} %Boards of width 3 mod 4 make clusters 1,3,6
  Boards of width $w=4k+3$ will produce exactly clusters of sizes $1,3,6$ at some point in their life cycles, starting with $k=1$. No other cluster sizes will be produced.
\end{prop}
\begin{proof}
  Clearly from $G_w(0)$ and $G_w(1)$ we get clusters of size $1,3$. We will get a cluster of size $6$ at $G_w(2k+1)$, which by~\cref{OddBoardFirstEdge} is the first time the board reaches an `edge.' We prove this by induction; it is true for base case $k=1$. Suppose it holds for $k$, we show it holds for $k+1$. $G_{4k+7}(2k+1) = 00\oplus G_{4k+3}(2k+1)00 = 00111\oplus A\oplus 11100$ by the induction hypothesis. Then $G_{4k+7}(2k+2)=01000\oplus B\oplus 00010$ and $G_{4k+7}(2k+3)=1110\oplus C\oplus 0111$. This is exactly a cluster of 6, at the correct time. The other direction holds by~\cref{cluster3mod4cor}.
\end{proof}

\addtocounter{prop}{1}
\begin{innerProp}
  $0\oplus G_{4k+2}(2s)=G_{4k+3}(2s)$, for all $s$.
\end{innerProp}
\begin{proof}
  We have by~\cref{onesSepByOddZeros} that $G_{4k+3}(2s)$ is of the form $b\cdot 0 \oplus (1 \oplus a_1\cdot 0 \oplus 1\oplus a_2\cdot 0\oplus\cdots \oplus 1\oplus a_n\cdot 0) \oplus 1 \oplus b$ where each $a_i\equiv 3 \Mod{4}$, and $b$ is odd. \\
  We proceed by induction. Suppose that for each even step $2s$, we have $G_{4k+3}(2s)=0\oplus G_{4k+2}(2s)$. The base case of $s=0$ is clear. \\
  Suppose at step $2s$ the board $G_{4k+3}(2s)$ is not near an edge, that is, $b>1$. Then the induction hypothesis clearly holds for $2s+2$. \\
  Suppose $G_{4k+3}(2s)=01\oplus A\oplus 10$ is near the `edge', that is, in the above format $b=1$. Then we have the following situation:
  \begin{align*}
    0100&\oplus A \oplus 0010 \\
    111&\oplus B \oplus 111 \\
    00&\oplus C \oplus 00
  \end{align*}
  where each new line is a new generation, starting with $G_{4k+3}(2s)$. Similarly, for this situation with $G_{4k+2}(2s)$ we get
  \begin{align*}
    100 &\oplus A\oplus 0010 \\
    11 &\oplus B\oplus 110 \\
    0 &\oplus C\oplus 00
  \end{align*}
  %\vspace*{-1cm}
\end{proof}
\begin{innerProp}
  $G_{4k+3}(s)\oplus 0=G_{4k+4}(s)$, for all $s$.
\end{innerProp}
\begin{proof}
  The proof proceeds identically to the above, except for the final step, where we show that both the step $2s+1$ and $2s+2$ proceed identically to how they do on boards of width $4k+3$, simply with a zero appended on the right: \\
  Starting with $G_{4k+4}(2s)$, where $2s$ happens to be near an edge, we get
  \begin{align*}
    0100&\oplus A \oplus 00100 \\
    111&\oplus B \oplus 1110 \\
    00&\oplus C \oplus 000 
  \end{align*}
\end{proof}

\begin{cor}\label{4mod4like3mod4}
  A board with width $w=4k+4$ has the same time to cycle, fuse length, and survival as the board with width $4k+3$. It has cluster sizes of $1,3$. 
\end{cor}
\begin{proof}
  This is immediate. The cluster sizes are the same, except that 6 never shows up. In $4k+3$ it shows up at the `edge,' however here that particular type of cluster is broken up by the zero appended to the right.
\end{proof}

\begin{cor}\label{2mod4Characterization}
  A board with width $w=4k+2$ annihilates or survives as the board with width $4k+3$ does, and has the same fuse length and time to cycle. Its clusters are $1,2,3$. 
\end{cor}
\begin{proof}
  The annihilation or survival follows from the associated proposition and~\cref{3mod4annihilateOnEven}. The fuse length and time to cycle follows~\cref{cycleOnEven3mod4}. The clusters are visible in the above proposition.
\end{proof}

We have been using the notion of the `edges' of the board without defining it, it has been based simply on the intuition of a torus seen as a rectangle with connected edges. However, the torus itself does not have true edges, so in what sense does it make sense to talk about them? We seed the board with an initial living `wall'. This `wall' we say is at the center of the torus, arbitrarily. The results would be the same if it were in any other location, though the 2-d display might look different. We call the opposite point on the torus the antipode, it is roughly what we previously called the `edges'. There is a minor issue --- there might not be an exact antipode, since we have a discrete quantity of cells. On an odd-width board, there is an odd amount of zeros on either side of the center, and so no single cell can be the antipode --- we call the two cells farthest from the center the antipodes. We will see that this is a reasonable definition.

Since the center is chosen arbitrarily, we can at any moment `refocus' and look at the board with a new center, a useful strategy. 
\begin{prop}\label{SymmetricAntipode}
  Let $w=4k+4$. Suppose the board reaches a state $G_w(s)=2^k-1\cdot 0\oplus 1\oplus a\cdot 0\oplus 1\oplus 2^k\cdot 0$, for some $k$ such that $2^{k+1}<W$ and $2^{k+1}<a$. Suppose $s$ is the first time this state has been seen, that is, there is no $t$ such that $t<s$ and $G_w(t)=G_w(s)$. Then there has been a step $t=2^k$ such that $G_w(t)=b\cdot 0 \oplus 1\oplus 0\cdot 2^{k+1}-1\cdot 1\oplus b+1\cdot 0$, with $t<s$, and $G_w(2s-t)=G_w(t)$. 
\end{prop}
\begin{proof} Recenter the board at the antipode, and note that now the board looks exactly like $b\cdot 0 \oplus 1\oplus 0\cdot 2^{k+1}-1\cdot 1\oplus b+1\cdot 0$, the final `$+1$' adjusting for even width. In effect, this is a previously seen state, except that it exists at the antipode. As seen centered at the antipode, it is a previously seen by~\cref{SingleWallPairPow2}, it occurred at exactly $t=2^k$. Since it took $s-t$ steps to get to this state again, but at the antipode, a further $s-t$ steps will bring us exactly back to $G_w(t)$.
\end{proof}
\begin{cor}\label{SymmetricAntipodeAll}
  For boards of width $w=4k+3$ we have the same, except it occurs when the board reaches a state with one less 0 on the right. For $w=4k+2$, it is again the same with one less 0 on both sides. 
\end{cor}
\begin{proof}
  This is evident by~\cref{4mod4like3mod4} and~\cref{2mod4Characterization}. 
\end{proof}
%\begin{cor}\label{Boards1001LikeRegular}
%  The above corollary implies that for boards of width $w=4k+3$, a board that is all dead except for $1001$ at the center behaves just like one with $10001$ at the center; it simply consistently has one %less 0.
%\end{cor}

%\begin{cor}
%  Let $w$ be a width such that $w\equiv 2,3,4 \Mod 4$. Suppose the board reaches a state as %described in the associated proposition, at step $s$, with $t$ the value such that %$G_w(t)=G_w(2s-t)$. Then $fuse(G_w)\leq t$. 
%\end{cor}
We will be using these boards with a gap of $2^k-1$ zeros in the center often, so for convenience we define the following.
\begin{dfn}
  Let $K_w(k)=a\cdot 0\oplus 1\oplus 2^k-1\cdot 0\oplus a\cdot 0$, with an adjustment for even boards. Let $K_w'(k)$ be $2^{k-1}-1\cdot 0\oplus a\oplus 2^{k-1}-1\cdot 0$, with a similar adjustment.
\end{dfn}

\begin{prop}\label{BetweenPows2}
  Let $w=2^k+2^{k+1}-1$, then for $k\geq 2$, $fuse(G_w)=2^{k-1}$ and $tc(G_w)=2^{k+1}-2^{k-1}$.
\end{prop}
\begin{proof}
  By~\cref{SingleWallPairPow2}, %$G_w(2^k)=a\cdot 0\oplus 1\oplus 2^{k+1}-1\cdot 0\oplus 1\oplus a\cdot 0$,
  $G_w(2^k)=K_w(k+1)$, with the number of 0s on the edges being $(2^k+2^{k+1}-1-(2^{k+1}-1)-2)/2=2^{k-1}-1$, that is, $G_2(2^k)=K_w(k+1)=K_w'(k)$ By~\cref{SymmetricAntipodeAll}, we have $G_w(2^{k-1})=G_w(2^k+2^{k-1})$. Since the boards prior to $G_w(2^k)$ follow a pattern described by~\cref{SingleWallPairPow2} without reaching the edges of the board, it is evident that this is in fact the first repeated board, so the fuse and time to cycle are as desired. 
\end{proof}
\begin{remark}
  The first state which will be repeated, $G_W(2^{k-1})$ is known by~\cref{SingleWallPairPow2}, so we have a well fleshed out description of these board-evolutions.
\end{remark}
Generalizing, we get the following.
\begin{cor}\label{Pow2PlusPow2}
  Let $w=2^k+2^\alpha-1$, for $k\geq 3$, $2\leq \alpha<k$. Then $fuse(G_w)=2^{\alpha-1}$, and $tc(G_w)=2^k-2^{\alpha-1}$.
\end{cor}
\begin{proof}
  Note that, by~\cref{SingleWallPairPow2}, $G_w(2^{k-1})=K_w(k)=K_w'(\alpha)$. We get the result by the same argument as in the associated proposition.
\end{proof}

We have examined boards of the form $2^k+2^\alpha$, now we switch to boards of form $2^k-2^\alpha$. 

\begin{comment}\begin{prop}
  Boards of size $w=2^n-5$ have $tc(G_w)=2^n-2$ and $fuse(G_w)=2$.
\end{prop}
\begin{proof} %2^n-5 - 2 - 2^{n-1}+1 = 2^{n-1} - 6  / 2 -> 2^{n-2} -3
  Consider the board after $2^{n-2}$ steps, by~\cref{Power2} we have that $G_w(2^{n-2})=2^{n-2}-3\cdot 0\oplus 1\oplus 2^{n-1}-1\cdot 0 \oplus 1\oplus 2^{n-2}-3\cdot 0$. By the same proposition and~\cref{OddBoardFirstEdge}, it will advance through the next $2^{n-2}-3$ steps without interacting with the edges. By the same logic as used in the proof of~\cref{Power2}, we conclude that $G_w(2^{n-1}-4) = 0\oplus 1\oplus 7\cdot 0\oplus 1\oplus 7 \cdot 0\oplus\cdots\oplus 1\oplus 7\cdot 0\oplus 1\oplus 0$, with enough repetitions to fill the board, that is, $((2^n-5)-3)/8 = (2^n-8)/8=2^{n-3}-1$ repititions. Then in two steps we get $000\oplus 1\oplus 000\oplus 1\oplus\cdots\oplus 000\oplus 1\oplus 000$, with $2^{n-2}-2$ 1s. And finally in two more steps this becomes $01\oplus 2^n-9\cdot 0 \oplus 10=G_w(2^{n-1})$. By~\cref{SymmetricAntipodeAll} we get the desired result. 
\end{proof} \end{comment}
\begin{prop}
  Boards of size $w=2^k-2^\alpha-1$ have $fuse(G_w)=2^{\alpha-1}$ and $tc(G_w)=2^k-2^{\alpha-1}$, for $2 \leq \alpha \leq k-3$, $k\geq 5$. 
\end{prop}
\begin{proof} %2^n-2^\alpha-1 - 2 - 2^{n-1}+1 = 2^{n-1} - 2^\alpha - 2  / 2 -> 2^{n-2} - 2^{\alpha-1} - 1
  %Examples of size 55, 59 are useful
  Consider the board after $2^{k-2}$ steps, by~\cref{Power2} we have $G_w(2^{k-2})=2^{k-2}-2^{\alpha-1}-1\cdot 0\oplus 1\oplus 2^{k-1}-1\cdot 0 \oplus 1\oplus 2^{k-2}-2^{\alpha-1}-1\cdot 0$. By~\cref{OddBoardFirstEdge} it will take $2^{k-2}-2^{\alpha-1}-1$ steps to reach the edge, so for this amount of time we can follow the format of boards of size $2^n-1$ as in~\cref{Power2}. That is, we have behavior akin to two boards of width $2^{k-2}+3$ for the next $2^{k-3}$ steps. At this point, we will have the board $G_W(2^{k-2}+2^{k-3}) = 2^{k-3}-2^{\alpha-1}-1\cdot 0 \oplus 1\oplus 2^{k-2}\cdot 0\oplus 1\oplus 2^{k-2}\cdot 0\oplus 1\oplus 2^{k-2}\cdot 0\oplus 1\oplus 2^{k-3}-2^{\alpha-1}-1\cdot 0$. Note that $2^{k-3}>2^{\alpha-1}+1$, since $\alpha$ is at most $k-3$, so this does in fact reach this point without reaching the edges.
  
  % For the next $2^{k-3}-2^{\alpha-1}$ steps we can treat each of these 1s as an independent board.
  A board of size $2^k-1$ will annihilate in $2^{k-1}$ steps. We are approaching that point, if we look at when our board reaches the edge, after a further $2^{k-3}-2^{\alpha-1}-1$ steps, we see that this is $G_w(2^{k-1}-2^{\alpha-1}-1)$, we are $(2^{\alpha-1}+1)$ steps away from that point. By~\cref{Pow2BelowPow2EvenSpacing} and the way we arrived at it, it is clear that $G_w(2^{k-1}-2^{\alpha-1})$ is an evenly spaced board padded with extra zeros, $G_w(2^{k-1}-2^{\alpha-1})=(2^\alpha-1\cdot 0\oplus 1\oplus2^\alpha-1\cdot 0\oplus 1\oplus\cdots\oplus2^\alpha-1\cdot 0\oplus 1)\oplus 2^\alpha-1\cdot 0$. 

  % It is fairly clear that a board evenly spaced like this will continue just as it did in the $2^n$ board, despite the loss of a living column on either side. The spacing is what makes it annihilate after a final $2^{\alpha-1}$ steps. Alternatively we could carry out the full analysis treating each column as a board of size $2^\alpha$. With either method, it is clear that
  By~\cref{EvenlySpacedPow2Annihilation}, the center portion of the board annihilates, $G_w(2^{k-1})=2^{\alpha-1}-1\cdot 0\oplus 1\oplus 2^k-2^{\alpha+1}-1\cdot 0\oplus 1\oplus 2^{\alpha-1}-1\cdot 0$. We also see that this is the first board which is a repetition of a previous one at the antipode, so by~\cref{SymmetricAntipodeAll} we get the desired result. 
  % 2^k-2^\alpha-1 - ( 2*(2^{\alpha-1}-1)+2 ) = 2^k-2^{\alpha+1}-1
  % 64 - 8 - 1 = 55; 64 - 16 - 1 = 47; correct
\end{proof}

We now have a fairly complete understanding of boards of size $2^k\pm 2^\alpha-1$, $1<\alpha<k$, and boards of size $2^k-1$. 



\begin{comment}
\begin{proof}
  Suppose there is a gap in the center, the board on an even step is of the form $A\oplus 1\oplus 0\cdot c \oplus 1\oplus A$, where $c\equiv 3\mod 4$, and $c>3$. This last condition is so that it is a true gap, otherwise it is the smallest possible space between living columns, so it would not make sense to call it a gap. We require that there are 1s on either side of the gap, otherwise we could examine a larger gap.
  Furthermore, suppose that $A$ is not all zeros.
  
  Then suppose that after two steps, we have a board with exactly 2 living columns, $a\cdot 0 \oplus 1\oplus c-4\cdot 0\oplus 1\oplus a\cdot 0$. Then consider what the $A$ must have been.
\end{comment}

\section{Boards of Width One Modulo Four}

We will start with boards of width $w=2^k+1$. However, in the analysis of those boards we will find a board which consists of $101$ padded by 0s on either side. So, first we will examine the evolution of these boards.
\begin{dfn}\label{101Boards}
  For an odd $w=2k+1$, let $H_w$ be a board evolution, $H_w:\N\rightarrow B_w$, defined by $H_w(0) = k-1\cdot 0 \oplus 101 \oplus k-1\cdot 0$. For even boards we add a zero on the right.
\end{dfn}

We develop a proposition similar to~\cref{Power2}.
% 5, 5+8, 5+8+16=5+24=29, 61=5+8+16+32=5
% 2, 6,   14, 30, 62,  
\begin{prop}\label{101BoardsBig} %5 + 8*(2^k-1) - (5+8*(2^{k-1}-1)) = 8((2^k-1) - (2^{k-1}-1)) = 8*2^{k-1}
  % 5 + 8*(2^k-1) = 2^{k+3}-3= (2^{k+2}-1)*2-1 = 2^{k+3}-3
  % (2^{k+2}-1)*2+1 = 2^{k+3}-1, subt 3 gets 2^{k+3}-4, and div by 4 gets 2^{k+1}-1
  Let $w(k)=5+8*(2^k-1)=2^{k+3}-3$ and $s(k)=2^{k+2}-2$, for $k\geq 0$. 
  $H_{w(k)}$ will annihilate in $s(k)$ steps, and on the previous step it will be alive at the edges, $H_{w(k)}(s(k)-1)=111:\oplus B\oplus 111$.
\end{prop}
\begin{proof}
  We take $w(0)$ a base case, $H_{w(0)}(0)=01010$ and $H_{w(0)}(1)=11011$, the next step is annihilation. Assume the hypothesis for values up to $n$, and prove it for $n+1$. \\
  By~\cref{PostAnnihilation} and the induction hypothesis, $H_{w(k+1)}(s(k))=4*2^k-1\cdot 0 \oplus 1\oplus w(k)\cdot 0 \oplus 1 \oplus 4*2^k-1$. Since $5+8*2^k-1 = 2^{k+3}-3$ we can re-write this as $2^{k+2}-1\cdot 0 \oplus 1\oplus (2^{k+2}-1)*2-1\cdot 0\oplus 1\oplus 2^{k+2}-1\cdot 0$. That is, we have what is very nearly $G_{2^{k+3}-1}(0)\oplus G_{2^{k+3}-1}(0)$, with one fewer 0 in the center. They each evolve without interacting for $2^{k+2}-2$ steps by~\cref{OddBoardFirstEdge}. We have a precise description of the board at this point by~\cref{Pow2BelowPow2EvenSpacing}, it consists of evenly spaced $1$s with a gap size of $3$. $H_{w(k+1)}(s(k)+2^{k+2}-2)=(0100)\cdot 2^{k+1}-1\oplus 01010 \oplus (0010)\cdot 2^{k+1}-1$. By applying the game of life rule, we get a board with $111$s on the edges, and by applying it again we get annihilation. Note that $s(k)+2^{k+2} = 2^{k+2}-2+2^{k+2}=s(k+1)$, as desired.
\end{proof}
\begin{cor}
  $H_{w(k)}$ produces only clusters of size 1,2,3,6 for $k>0$. We do this by induction starting with $k=1$ as a base case, and proceeding as above. After the first $s(k)$ steps, where we have this by induction, it proceeds as $G_a$ for a $a\equiv 3 \Mod 4$, which produces cluster sizes of 1 and 3. We get clusters of size 2 and 6 on the last two steps.
\end{cor}

\begin{prop}
  A board of width $w=2^k+1$ has $tc(G_w)=3*2^{k-1}-3$, $fuse(G_w)=2^{k-1}+1$, and seen cluster sizes 1, 2, 3, and 6.
\end{prop}
\begin{proof}
  As per~\cref{Power2} we have $G_w(2^{k-1}) = 1\oplus 2^k-1\cdot 0\oplus 1$. The next step is $G_w(2^{k-1}+1)=01\oplus 2^k-3\cdot 0\oplus 10$. Recentering at the antipode, we see a $1001$ padded by 0s. We have seen this form of board before, it occurs partway through the evolution of a board of size $2^k+3$, at step $G_{w+2}(2^{k-1})$, by~\cref{Pow2PlusPow2}. On these boards, evolution continues without interacting at the center all the way up to the point at which it cycles, just as it reached this point without interacting at the edges. At that point, the board is of the form $10001$, padded by 0s, this happens in $tc(G_{w+2})-2^{k-1}=2^{k-1}-2$ steps. Since we have that no interaction occured at the center, it is clear that in our case, we will get to the same result, but with 2 fewer 0s at the center, that is, $G_w(2^{k-1}+1+2^{k-1}-2) = G_w(2^k-1)=2^{k-1}-1\cdot 0 \oplus 101\oplus 2^{k-1}-1\cdot 0$. \\
  Before continuing, we note that we have seen clusters of size 1,2,3 in the steps up through $G_w(2^{k-1}+1)$, that in the very next step we see a cluster of size 6, and that in the rest of the discussed steps the board has followed the pattern of a board of size $2^k+3$, so those have produced only clusters of size 1,3,6. \\
  Now $G_w(2^k-1)=H_w(0)$. The next $2^{k-1}-2$ steps lead to the annihilation of $H_{w-4}$ by~\cref{101BoardsBig}, and this proposition gets the preconditions for~\cref{PostAnnihilation} which gets us that $G_w(2^k+2^{k-1}-3)=G_w(3*2^{k-1}-3)=01\oplus 2^k-3\cdot 0\oplus 10$, a state which we have seen before at step $2^{k-1}+1$. It is clear that no intermediate state has been repeated, so the time to cycle and fuse $3*2^{k-1}-3$ and $2^{k-1}+1$ as 
\end{proof}

The board of width 61 interests us, as it annihilates. In its evolution, we get a $1111$ cluster at the antipode, with the rest of the board dead. We examine boards of this form, recentering to the antipode, and find that after a few steps, they form what can be viewed of as a pair of boards with $11101$ and its reflection. So we examine boards with this at the center as a starting state:
\begin{dfn}\label{11101Boards}
  For an odd $w=2k+1$, let $F_w$ be a board evolution, $F_w:\N\rightarrow B_w$, defined by $F_w(0) = k-2\cdot 0 \oplus 11101 \oplus k-2\cdot 0$. For even boards we add a zero on the right.
\end{dfn}

As is becoming our modus operandi, we form a proposition reminiscent of~\cref{Power2}. However these boards do not annihilate as easily, instead they repeat.
\begin{prop}\label{11101BoardsBig}
  Let $w(k)=2^k+5$, and $s(k)=2^{k-1}$, for $k\geq 3$. Then $F_{w(k)}(s(k)-1)=010001111111\oplus (2^k-8)/4\cdot 0111\oplus 0$, which notably implies that
  $F_{w(k)}(s(k))=11101\oplus w(k)-6\cdot 0\oplus 1$. Also, these states are reached without interacting at the edges.
\end{prop}
\begin{proof}
  We take $k=3$ as a base case, it is easy to see. Then by induction suppose it works for up to $k$, we show it works for $k+1$. We have that $F_{w(k+1)}(s(k))=2^{k-1}\cdot 0\oplus 11101\oplus 2^k-1\cdot 0\oplus 1\oplus 2^{k-1}\cdot 0 = F_{w(k)}(0)\oplus G_{2^k}(0)$. Then by our induction hypothesis,~\cref{Power2}, and~\cref{EvenlySpacedPow2Annihilation}, $F_{w(k+1)}(s(k+1)-1) = 010001111111\oplus (2^{k+1}-8)/4\cdot 0111\oplus 0$, the $1110$ blocks from the right part combine seamlessly with those on the left. By an application of the game of life rule we get that the next step is also as desired.
\end{proof}
Notice that for such $w$, on every fourth step, $F_w$ looks much like $G_w$, for the first $w/2$ steps. We formalize this.
\begin{cor}\label{11101LikeReg}
  For some $k\geq 3$, let $s=4s'$, and $s<=2^{k-1}$. Then $F_{w(k)}(s)$ is precisely $G_{w(k)-4}(s)$ with $1110$ inserted directly left of the leftmost $1$. 
\end{cor}
\begin{proof}
  We proceed by induction, first on $k$ and then on $s$. Let $k=3$, then we only test $s=0,4$, it is clear in both these cases. Then by induction suppose that this is true up to $k$, and consider it for $k+1$. For $s=0$ this is clear. %Moreover, it is clear for $s\leq 2^\alpha$, with $2\leq \alpha\leq k-1$ by the associated proposition. Then for any $s=4s'$, let $\alpha$ be the largest integer such that $2^\alpha\leq 4s'$. If $2^\alpha=4s'$ we're done. Otherwise, we have that
  Since we have this up to $k$, and there in no interaction at the edge for the $s$ under consideration, we also have it up to $s=2^{k-1}$ for $w=w(k+1)$, since the board is the same as with $w(k)$ simply padded with 0s. Then suppose $s>2^{k-1}$. We have that $F_{w(k+1)}(2^{k-1})=F_{w(k)}(0)\oplus G_{2^k}(0)$. Then let $r=s-2^{k-1}$, we have that the proposition holds for $F_{w(k)}(r)$ by the inductive hypothesis. Then, clearly, inserting $1110$ left of the leftmost living cell in  $G_{w(k+1)-4}(s)$ gets $F_{w(k)}(r)\oplus G_{2^k}(r)=F_{w(k+1)}(s)$ as desired.
\end{proof}

We proceed to boards with a cluster of 4 as the starting point.
\begin{dfn}\label{initCenterClusterBoards}
  We define a family of maps $G_w:\N\times \N\rightarrow \B_w$ as follows.
  \[ G_w(n,s)=\begin{cases}
      C^s((w-n)/2\cdot 0\oplus n\cdot 1\oplus (w-n)/2\cdot 0) & \text{if } n\equiv w \Mod(2) \\
      C^s((w-n-1)/2\cdot 0\oplus n\cdot 1\oplus (w-n-1)/2 + 1\cdot 0) & \text{if } n\not\equiv w \Mod(2) \end{cases} \]
  It is undefined if $n>w$. For a fixed $n\leq w$, $G_w(n, s)$ is a board evolution with a living cluster of width $n$ in the center. Notably, the $G_w(s)$ defined in~\cref{mainBoard1Center} is simply convenient notation for $G_w(1,s)$. 
\end{dfn}

\begin{prop}\label{1111BoardsBig}
  For $k\geq 6$, the following state is reached without interaction at the edges:
  \begin{align*}
    G_{2^k+4}(4, 2^{k-1}-1)=01000 &\oplus 7\cdot 1\oplus 0111\cdot 5\oplus (0000 \oplus 0111\cdot 7)\cdot 2^{k-6}-1\oplus \\
    000 &\oplus (0111\cdot 7\oplus 0000) \cdot 2^{k-6}-1\oplus 0111\cdot 5\oplus 0111111100010
  \end{align*}
  Clearly this implies that $$G_{2^k+4}(4, 2^{k-1}) = 11101\oplus 27\cdot 0\oplus (10001\oplus 27\cdot 0)\cdot 2^{k-6}-1\oplus 1001\oplus (27\cdot 0\oplus 10001)\cdot 2^{k-6}-1\oplus 27\cdot 0\oplus 10111$$
\end{prop}
\begin{proof}
  First, the lack of interaction at the edges is clear by a simple extension of~\cref{OddBoardFirstEdge} (generalization at~\cref{???}.) We proceed by induction, and take $k=6$ as a base case, $G_{68}(4, 31)$ does follow the proper form. Suppose the proposition holds for $k$, we prove it for $k+1$. Clearly, $G_{2^{k+1}+4}(4,2^{k-1})=2^{k-1}\cdot 0\oplus G_{2^k+4}(4,2^{k-1})\oplus 2^{k-1}\cdot 0$. By~\cref{SymmetricAntipodeAll} we see that the $1001$ parts behaves similarly to the $10001$ parts, having one less zero in the center simply makes every successor state have one less zero. Then, by this fact and repeated applications of~\cref{Power2}, and~\cref{11101BoardsBig}; or simply applications of the game of life rule, we see that $G_{2^{k+1}+4}(4,2^{k-1}+13)$ will be the first time the isolated segments begin interacting, they will cause the center to annihilate, and in another two steps we will get another smaller `annihilation event' and end up with $G_{2^{k+1}+4}(4,2^{k-1}+16)=2^{k-1}-16\cdot 0\oplus 11101\oplus 27\cdot 0\oplus 1\oplus 2^k-30\cdot 0\oplus 1\oplus 27\cdot 0\oplus 10111\oplus 2^{k-1}-16$.

  Again, by~\cref{11101BoardsBig} and~\cref{Power2} or applications of the game of life rule, we get that $G_{2^{k+1}+4}(4,2^{k-1}+31)=2^{k-1}-31\cdot 0 \oplus 1000\oplus 7\cdot 1\oplus 0111\cdot 5\oplus 00000\oplus 1110\cdot 7\oplus 2^k-61\cdot 0\dots$, where the $\dots$ represent that the rest of the board is symmetric, it is already described. For $k=7$ this is exactly the desired state, we are done. We use $k=7$ as a base case also.

  Notably, the successor of the state described above, $G_{2^{k+1}+4}(4,2^{k-1}+32)$ is similar to having two boards $B=G_{2^6+4}(4, 32)$, with zeros between and around them. The difference is that in $B$, there is a spare $0111$ on the right, or a $1110$ on the left, depending on which symmetric half of our board is looked at. However, as described in~\cref{11101LikeReg}, this doesn't change how it interacts with the rest of $B$ at all, as long as the first interaction is at a step which is divisible by 4. The other difference is that in the very center of $B$ there is one less 0 than in this board. However, as noted previously, $1001$ and $10001$ behave the same way. The sole difference is that that one extra $0$ will stay throughout. Therefore, we have $G_{2^{k+1}+4}(4,2^{k-1}+2^5)\approx 2^{k-1}-2^5\cdot 0\oplus G_{2^k+4}(4,2^5)\oplus 2^k-(2^6-2)\cdot 0\oplus G_{2^6+4}(4,2^5)\oplus 2^{k-1}-2^5\cdot 0$. Then by the induction hypothesis, in a further $2^{k-1}-2^5$ steps the state will be $G_{2^{k+1}+4}(4,2^k)\approx G_{2^k+4}(4,2^{k-1})\oplus 00\oplus G_{2^k+4}(4,2^{k-1})$, with the approximation meaning that this state is exactly what is desired. 
\end{proof}
\end{document}